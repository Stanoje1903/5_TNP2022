\documentclass{beamer}
\usepackage{beamerthemeshadow}
\usepackage{graphicx}
\usepackage{color}
\usepackage[utf8]{inputenc}
\usepackage{hyperref}
\usepackage[flushleft]{threeparttable}
\usepackage{lipsum}
\usepackage{caption}
\usepackage{tabularx}
\usepackage{url}
\usepackage[english,serbian]{babel}
\definecolor{royalblue}{RGB}{26, 22, 130}
\setbeamercolor{structure}{fg=royalblue}


\title{Tehničko i naučno pisanje}
\subtitle{-- Stiven Hoking --}
\author{ Ivana Milenković\\ \and Lazar Rajčić\\ \and Anđela Spasić\\ \and Nikola Stanojević }
\institute{Matematički fakultet\\Univerzitet u Beogradu}
\date{
	\footnotesize{Beograd, 2022.}	
}

\begin {document}
\begin{frame}
	\thispagestyle{empty}
	\titlepage
\end{frame}

\addtocounter{framenumber}{-1}

\begin{frame}[fragile]\frametitle{Literatura}
	\begin{itemize} \fontsize{9}{6}\selectfont
		\item Zasnovano na:\\

	\item Stephen Hawking
	     \\ (\url{https://www.famousscientists.org/stephen-hawking/})
        \item  Stephen Hawking - British physicist
           \\ (\url{ https://www.britannica.com/biography/Stephen-Hawking})
        \item Stephen Hawking's Best Books: Black Holes, Multiverses and Singularities
           \\(\url{https://www.space.com/39987-stephen-hawking-best-books.html})
        \item Hawking's medals and awards
            \\(\url{ https://onlineonly.christies.com/s/shoulders-giants-newton-darwin-einstein-hawking/hawkings-medals-awards-50/62110})
       \item  Stephen Hawking
            \\(\url{ https://www.hawking.org.uk/biography})
        \item Stephen William Hawking CH CBE. 8 January 1942—14 March 2018
           \\ (\url{https://royalsocietypublishing.org/doi/10.1098/rsbm.2019.0001})
        \item J. Laski and W. Stanley. \emph{Software Verification and Analysis}. Springer- Verlag, London, 2009.
     \fontsize{6pt}{1pt}\selectfont
	\end{itemize}
\end{frame}

\begin{frame} \fontsize{9}{6}\selectfont
	\frametitle{Pregled} % Table of contents slide, comment this block out to remove it
	\tableofcontents[hidesubsections] 
\end{frame}
\section{Ko je Stiven Hoking?}

\begin{frame}[fragile]\frametitle{Ko je Stiven Hoking?}
	\begin{itemize} \fontsize{9}{6}\selectfont	
			\item  Stiven Hoking - engleski teoretski fizičar i kozmolog. 
  \item Jedan od najvećih naučnika svog vremena.
  \item Njegovi doprinosi fizici se uglavnom nalaze u domenu našeg poznavanja crnih rupa.
\begin{figure}[h!]
  \centering
  \includegraphics[width=0.3\textwidth]{Hoking,PreDijagnoze.jpg}
  \captionsetup{font=small}{Slika1: Hoking, na dodeli diploma 1960.}
  \label{fig:Hoking,PreDijagnoze}
  \end{figure}
	\end{itemize}
\end{frame}

\section{Život}
\begin{frame}[fragile]\frametitle{Rani život (Pre dijagnoze i dijagnoza)}
\begin{itemize}	 \fontsize{9}{6}\selectfont
		
\item  Hoking se rodio 8. Januara 1942. godine, tačno na tristotu godišnjicu smrti Galilea Galileja.
\item  U ranijim školskim godinama je dobio nadimak Ajnštajn.
\item  Osnovne studije je završio 1962. godine na Oksfordu nakon čega prelazi na Kembridž univerzitet.
\item  Nedugo posle 21.-og rođendana saznaje da ima amiotrofičku lateralnu sklerozu (eng. Amyotrophic lateral sclerosis) skraćeno ALS.
\end{itemize}
\end{frame}

\section{Život nakon dijagnoze}

\begin{frame}[fragile]\frametitle{Život nakon dijagnoze}
	\begin{itemize} \fontsize{9}{6}\selectfont	
		\item  Ne uspeva da pohađa pod Fredom Hojlom kao sto se tada nadao, tako da ih je nastavio pod Denisom Škiamom. 
            \item Škiam ga je upoznao sa Rodžerom Penrouzom 1965. godine. Ova godina je jako znacajna za Hokinga zbog mnogih razloga.
		\item  U narednim godinama njegova bolest postaje sve ozbiljnija.
		\item  Preminuo je u udobnosti svog doma, 14. marta 2018. godine.
\begin{figure}[h!] 
\begin{center}
\includegraphics[width=0.3\textwidth]{StivenHoking.jpeg}
\end{center} 
 \captionsetup{font=small}{Slika2: Stiven Hoking i njegova žena} 
\label{fig:StivenHoking}
\end{figure}
\end{itemize}
\end{frame}


\section{Kariera}

\begin{frame}[fragile]\frametitle{Kariera}
\begin{itemize}	 \fontsize{9}{6}\selectfont
		
\item  Primarno je radio na poljima opšte relativnosti, posebno se interesujući za sferu crnih rupa.
\item Teorija povezanosti velikog praska i relativiteta
\item Teorija malih crnih rupa
\item Hokingova radijacija 
\end{itemize}
\end{frame}

\begin{frame}[fragile]\frametitle{Kariera}
\begin{itemize}	 \fontsize{9}{6}\selectfont	
\item Bio je član Kraljevskog društva (eng. Fellow of the Royal Society-FRS)
 \item Doživotni član Papeške akademije nauka
 \item Dobitnik predsedničke medalje o slobodi.
 \item Profesor gravitacione fizike i Lukasovski profesor matematike (eng. Cambridges Lucasian professorship of mathematics)
 \item Komandant Ordena Britanske imperije  i Pratioc časti (eng. Companion of Honor) \item Dobitnik Koplei medalje (eng. Coplei medal) i Američke predsedničke medalje slobode.
\end{itemize}
\end{frame}

\subsection{Njegove najprodavanije knjige}

\begin{frame}[fragile]\frametitle{ Njegove najprodavanije knjige}
	\begin{itemize}	\fontsize{9}{6}\selectfont	
		\item Neke od knjiga koje je napisao a koje su se našle na listi najprodavajih su:
		\begin{itemize}\fontsize{9}{6}\selectfont
 \item Kratka povest vremena (1988) - upućena osobama koje nisu naučnici i proučava osnove univerzuma, kako je nastao i njegov mogući kraj
 \item Crne rupe i bebe vaseljene (1993) – skup Hokingovih eseja, od naučnih do privatnih
 \item Kosmos u orahovoj ljusci (2001) – nastavak na knjigu “Kratka povest vremena”, objašnjava pojmove kao što su supergravitacija, kvantna fizika, itd.
 \item Na plećima divova (2002) – pogled kroz radove Kopernika, Galilea, Keplera, Njutona i Ajnštajna
 \item Kraća povest vremena (2005) – dodatak na “Kratku povest vremena” i “Kosmos u orahovoj ljusci”
 \item Bog je stvorio cele brojeve: Matematički prodori koji su promenili istoriju (2005) –  prikazuje najvažnije delove istorije matematike, kao i biografije važnih matematičara
 \item Velika zamisao (2010) – opisuje poreklo univerzuma i prirodu realnosti, jedna od tema knjige su paralelni univerzumi
 \item Kratka istorija mog života (2013) – njegov memoar
\end{itemize}
\end{itemize}
\end{frame}

\subsection{Dostignuća}

\begin{frame}[fragile]\frametitle{Dostignuća}
\vspace{1cm}
\begin{table}[h!]
\centering
\caption{Nagrade koje je Stiven Hoking dobio tokom svog života.} 
\medskip
\resizebox{\columnwidth}{!}{%
\begin{tabular}{|c|c|c|}
\hline
Godina & Nagrada & Razlog \\ \hline
1975 & \begin{tabular}[c]{@{}c@{}}Edingtonova\\  medalja\end{tabular} & Za otkrića iz 1970. godine \\ \hline
1976 & \begin{tabular}[c]{@{}c@{}}Medalja Džejms Klark fizičkog\\  instituta\end{tabular} & \begin{tabular}[c]{@{}c@{}}Za izvanredne rano-karijerske doprinose\\ teoretskoj fizici\end{tabular} \\ \hline
1978 & \begin{tabular}[c]{@{}c@{}}Nagrada Albert \\ Ajnštajn\end{tabular} & Za uspeh u prirodnim naukama \\ \hline
1979 & Medalja Društva Alberta Ajnštajna & \begin{tabular}[c]{@{}c@{}}Za naučne radove povezane sa Ajnštajnom\\ (Hoking je bio prvi primalac)\end{tabular} \\ \hline
1985 & \begin{tabular}[c]{@{}c@{}}Zlatna medalja kraljevskog \\ astronomskog društva\end{tabular} & Za doprinose astronomiji \\ \hline
1989 & Nagrada Britanika & \begin{tabular}[c]{@{}c@{}}Za širenje\\  znanja\end{tabular} \\ \hline
1999 & \begin{tabular}[c]{@{}c@{}}Medalja društva umetničkih proizvođača\\ i trgovine\end{tabular} & \begin{tabular}[c]{@{}c@{}}Za činjenje fizike dostupnom i \\ razumljivijom\end{tabular} \\ \hline
\end{tabular}%
}
\end{table}
\end{frame}

\section{Zakljucak}
\begin{frame}[fragile]\frametitle{Zakljucak}
\begin{itemize}	 \fontsize{9}{6}\selectfont	
\item  Da nije bilo Stivena Hokinga ne bismo znali ni pola onoga što znamo o crnim rupama. Njegovi doprinosi fizici su nas daleko pogurali u našim daljim saznanjima, dok njegove knjige pomažu osobama koje se ne bave naukom da je malo bolje upoznaju. Bilo da je zbog njegove borbe protiv ALS-a ili zbog njegovih naučnih radova, Stiven Hoking je inspiracija mnogih ljudi čak i nakon smrti. Zahvalni smo njegovom intelektu, želji da podeli znanje i hrabrosti i upornosti protiv sopstvenih ograničenja. 
\end{itemize}
\end{frame}
\end{document}