\documentclass{beamer}
\usepackage{beamerthemeshadow}
\usepackage{graphicx}
\usepackage{color}
\usepackage[utf8]{inputenc}
\usepackage{hyperref}
\usepackage[flushleft]{threeparttable}
\usepackage{lipsum}
\usepackage{caption}
\usepackage{tabularx}
\usepackage{url}
\usepackage[english,serbian]{babel}
\definecolor{royalblue}{RGB}{26, 22, 130}
\setbeamercolor{structure}{fg=royalblue}


\title{Tehničko i naučno pisanje}
\subtitle{-- Stiven Hoking --}
\author{ Ivana Milenković\\ \and Lazar Rajčić\\ \and Anđela Spasić\\ \and Nikola Stanojević }
\institute{Matematički fakultet\\Univerzitet u Beogradu}
\date{
	\footnotesize{Beograd, 2022.}	
}

\begin {document}
\begin{frame}
	\thispagestyle{empty}
	\titlepage
\end{frame}

\addtocounter{framenumber}{-1}

\begin{frame} \fontsize{9}{6}\selectfont
	\frametitle{Pregled}
	\tableofcontents[hidesubsections] 
\end{frame}
\section{Ko je Stiven Hoking?}

\begin{frame}[fragile]\frametitle{Ko je Stiven Hoking?}
	\begin{itemize} \fontsize{9}{6}\selectfont	
		\item  Stiven Hoking je bio engleski teoretski fizičar i kozmolog. Zbog njegovih doprinosa fizici smatran je za jednog od najvećih naučnika svog vremena. Njegovi doprinosi fizici se uglavnom nalaze u domenu našeg poznavanja crnih rupa. Zbog ovoga smatramo da je jako važno približiti njegova dostignuća ljudima.
\begin{figure}[h!]
  \centering
  \includegraphics[width=0.3\textwidth]{Hoking,PreDijagnoze.jpg}
  \captionsetup{font=small}{Slika1: Hoking, na dodeli diploma 1960.}
  \label{fig:Hoking,PreDijagnoze}
  \end{figure}
	\end{itemize}
\end{frame}

\section{Život}
\begin{frame}[fragile]\frametitle{Rani život (Pre dijagnoze i dijagnoza)}
\begin{itemize}	 \fontsize{9}{6}\selectfont
		
\item  Hoking se rodio 8. Januara 1942. godine, tačno na tristotu godišnjicu smrti Galilea Galileja.
\item  U ranijim školskim godinama je dobio nadimak Ajnštajn.
\item  Zbog primedbi svog oca nastavio je da studira fiziku, a ne matematiku na Oksfordu. Osnovne studije je završio 1962. godine nakon čega prelazi na Kembridž univerzitet.
\end{itemize}
\end{frame}
\begin{frame}[fragile]\frametitle{Život nakon dijagnoze}
	\begin{itemize} \fontsize{9}{6}\selectfont	 
		\item  Škiam ga je upoznao sa Rodžerom Penrouzom sa kojim će kasnije sarađivati na jednom od svojih najvažnijih radova.
		\item  1965. godine neposredno nakon dobijanja doktorata Hoking stupa u bračnu zajednicu sa Džejn Vajld i prijavljuje se za istraživačku stipendiju.
		\item  U narednim godinama njegova bolest postaje sve ozbiljnija. Dolazi do tog nivoa da gubi mogućnost hoda kao i govora. Ovo uspeva da prevaziđe uz pomoc kolica i mašine koja sintetiše ljudski govor. 
		\item  Preminuo je u udobnosti svog doma, 14. marta 2018. godine. 
\begin{figure}[h!] 
\begin{center}
\includegraphics[width=0.3\textwidth]{StivenHoking.jpeg}
\end{center} 
 \captionsetup{font=small}{Slika2: Stiven Hoking i njegova žena} 
\label{fig:StivenHoking}
\end{figure}
\end{itemize}
\end{frame}


\section{Kariera}

\begin{frame}[fragile]\frametitle{Kariera}
\begin{itemize}	 \fontsize{9}{6}\selectfont
\item  Zvaničan početak njegove karijere označava godina 1965. Primarno je radio na poljima opšte relativnosti, posebno se interesujući za sferu crnih rupa.
\item Hoking je, 1970. godine zajedno sa Rodžerom Penrouzom dožao do veze između teorije relativiteta i velikog praska. Godinu dana kasnije dolazi do teorije malih crnih rupa, kojom je nađena veza između kvantne i klasične fizike, čije oktriće spada u jedno od Stivenovih najpoznatijih dela. Tri godine kasnije iznosi teoriju Hokingove radijacije. 
\end{itemize}
\end{frame}

\section{Njegove najprodavanije knjige}

\begin{frame}[fragile]\frametitle{ Njegove najprodavanije knjige}
	\begin{itemize}	\fontsize{9}{6}\selectfont	
		\item Neke od knjiga koje je napisao a koje su se našle na listi najprodavajih su:
		\begin{itemize}\fontsize{9}{6}\selectfont
 \item Kratka povest vremena (1988) - upućena osobama koje nisu naučnici i proučava osnove univerzuma, kako je nastao i njegov mogući kraj
 \item Crne rupe i bebe vaseljene (1993) – skup Hokingovih eseja, od naučnih do privatnih
 \item Kosmos u orahovoj ljusci (2001) – nastavak na knjigu “Kratka povest vremena”, objašnjava pojmove kao što su supergravitacija, kvantna fizika, itd.
 \item Na plećima divova (2002) – pogled kroz radove Kopernika, Galilea, Keplera, Njutona i Ajnštajna
 \item Kraća povest vremena (2005) – dodatak na “Kratku povest vremena” i “Kosmos u orahovoj ljusci”
 \item Bog je stvorio cele brojeve: Matematički prodori koji su promenili istoriju (2005) –  prikazuje najvažnije delove istorije matematike, kao i biografije važnih matematičara
 \item Velika zamisao (2010) – opisuje poreklo univerzuma i prirodu realnosti, jedna od tema knjige su paralelni univerzumi
 \item Kratka istorija mog života (2013) – njegov memoar
\end{itemize}
\end{itemize}
\end{frame}